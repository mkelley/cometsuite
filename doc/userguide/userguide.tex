\documentclass[12pt,letterpaper]{article}
\usepackage[pdftex]{graphicx}
\usepackage{amssymb}
\usepackage{amsmath}
\usepackage{alltt}
\usepackage{listings}
\usepackage{natbib}
\bibpunct{(}{)}{;}{a}{}{,}
\usepackage[pdftex,colorlinks,bookmarks,citecolor=blue,%
  bookmarksnumbered,bookmarksopen=true,breaklinks,%
  hyperfootnotes=false]{hyperref}
\usepackage[width=6.5in,height=9in]{geometry}
\usepackage{setspace}

\newcommand\micron{\mbox{$\mu$m}}%
\newcommand\arcdeg{\mbox{$^\circ$}}%
\newcommand\arcmin{\mbox{$^\prime$}}%
\newcommand\arcsec{\mbox{$^{\prime\prime}$}}%
\newcommand\phm[1]{\phantom{#1}}%
\newcommand\nodata{ ~$\cdots$~ }%
\newcommand\inv{$^{-1}$}
\newcommand\mjysr{MJy~sr$^{-1}$}
\newcommand\invrho{~$\rho^{-1}$}
\newcommand\kms{km~s$^{-1}$}
\newcommand\wcm{W~cm$^{-2}$~\micron$^{-1}$}
\newcommand\gcm{g~cm$^{-3}$}
\newcommand\mks{J~K$^{-1}$~m$^{-2}$~s$^{-1/2}$}
\newcommand\degr{\arcdeg}%
\newcommand\Sun{\sun}% Sun symbol, "S"
\newcommand\sun{\odot}%
\newcommand\sst{\textit{Spitzer Space Telescope}}
\newcommand\spitzer{\textit{Spitzer}}
\newcommand\iso{\textit{ISO}}
\newcommand\isolong{\textit{Infrared Space Observatory}}
\newcommand\rundynamics{RunDynamics}
\newcommand\cs{CometSuite}
\newcommand\bvec[1]{\boldsymbol{#1}}


\begin{document}
\title{CometSuite}
\author{Michael S. Kelley\\
Department of Astronomy\\
University of Maryland, College Park\\
{\small msk @t astro.umd.edu}}

\date{03 Apr 2009}
\maketitle
\tableofcontents

\clearpage

\section{Introduction}
\cs{} is a set of C/C++ and IDL programs primarily designed for
modeling and analysing comet dust dynamics.  The \cs{}:~IDL programs
are used to determine comet, planet, and \spitzer{} positions in the
solar system; and to compute comet ephemerides for an arbitrary
observer.  The \cs{}:~\rundynamics{} programs simulate comet dust
dynamics; create syndyne files suitable for use in SAOImage/DS9;
create projected images of a comet simulation; and to inspect and
verify \rundynamics{} data files.

This document is a description of the theory and implementation of the
\cs{} dynamical model, dust grain model, and synthetic imager.

%%%%%%%%%%%%%%%%%%%%%%%%%%%%%%%%%%%%%%%%%%%%%%%%%%%%%%%%%%%%%%%%%%%%%%%%%%%%%%%%
% dust dynamics
%%%%%%%%%%%%%%%%%%%%%%%%%%%%%%%%%%%%%%%%%%%%%%%%%%%%%%%%%%%%%%%%%%%%%%%%%%%%%%%%
\section{Dust Dynamics}\label{sec:dynamics}

\lstset{language=csh,basicstyle=\normalsize\ttfamily\singlespacing,
  showstringspaces=false,columns=fullflexible}

\subsection{The integrator RADAU15}


\subsection{Ejection from the comet nucleus}

Start with the comet's heliocentric velocity and position (and
HORIZONS).


location on the nucleus


\subsubsection{Grain speed}

size dependence

location dependence

\subsubsection{Grain direction}

location dependence

\subsection{Dust in solar orbit}

beta parameter

\subsection{Planetary perturbations}

gravitational forces from the planets

including/excluding planets

Ceres, Vesta, Juno?

\subsection{Relativistic corrections}


%%%%%%%%%%%%%%%%%%%%%%%%%%%%%%%%%%%%%%%%%%%%%%%%%%%%%%%%%%%%%%%%%%%%%%%%%%%%%%%%
% dust grains
%%%%%%%%%%%%%%%%%%%%%%%%%%%%%%%%%%%%%%%%%%%%%%%%%%%%%%%%%%%%%%%%%%%%%%%%%%%%%%%%
\section{Dust Grains}\label{sec:grains}

\subsection{Grain sizes}

the primary dust parameter

size distributions

\subsection{Grain composition}

carbon and silicates

indices of refraction

other compositions?  organics?  ices?

\subsection{Grain structure}

fluffy aggregates

monomer sizes

fractal dimension

effective medium theory

Maxwell Garnet vs Bruggeman


\subsection{Computing $\beta$}



%%%%%%%%%%%%%%%%%%%%%%%%%%%%%%%%%%%%%%%%%%%%%%%%%%%%%%%%%%%%%%%%%%%%%%%%%%%%%%%%
% comet dust production
%%%%%%%%%%%%%%%%%%%%%%%%%%%%%%%%%%%%%%%%%%%%%%%%%%%%%%%%%%%%%%%%%%%%%%%%%%%%%%%%
\section{Comet dust production}\label{sec:dustproduction}

As a function of time: uniform

Grain weighting in the synthetic imager

Restrict by beta/radius/age range

As a function of heliocentric distance: rh cutoff




\subsection{}


%%%%%%%%%%%%%%%%%%%%%%%%%%%%%%%%%%%%%%%%%%%%%%%%%%%%%%%%%%%%%%%%%%%%%%
\begin{thebibliography}{}
\bibitem[Standish(2006)]{standish06} Standish, E.~M.\ 2006, JPL IOM,
  343.R-06-002

\bibitem[Folkner et al.(2008)]{folkner08} Folkner, W.~M., Williams,
  J.~G., \& Boggs, D.~H.\ 2008, JPL IOM, 343R-08-003

\end{thebibliography}

\end{document}
